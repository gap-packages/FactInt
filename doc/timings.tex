%%%%%%%%%%%%%%%%%%%%%%%%%%%%%%%%%%%%%%%%%%%%%%%%%%%%%%%%%%%%%%%%%%%%%%%%%
%%
%W  timings.tex             FactInt documentation             Stefan Kohl
%%
%H  @(#)$Id$
%%
%%%%%%%%%%%%%%%%%%%%%%%%%%%%%%%%%%%%%%%%%%%%%%%%%%%%%%%%%%%%%%%%%%%%%%%%%

\Chapter{How much Time does a Factorization take ?}

%%%%%%%%%%%%%%%%%%%%%%%%%%%%%%%%%%%%%%%%%%%%%%%%%%%%%%%%%%%%%%%%%%%%%%%%%
\Section{The General Factorization Routine}

A few words in advance: In general, it is not possible to give a precise
prediction for the CPU time needed for factoring a given integer.
This time depends heavily on the sizes of the factors of the given number
and on some other properties which cannot be tested before actually
doing the factorization.
But nevertheless, rough runtime estimates can be given for numbers
with factors of given orders of magnitude.

After casting out the small and other ``easy'' factors -- which should
not take more than at most a few minutes for numbers of
``reasonable'' size -- the general factorization routine uses first ECM
(see~"FactorsECM") for finding factors very roughly up to the third root
of the remaining composite and then the MPQS (see~"FactorsMPQS") for
doing the ``rest'' of the work. The latter is often the most
time-consuming part.

In the sequel, some timings for ECM and for the MPQS are given.
These methods are by far the most important ones with respect to
runtime statistics (the $p \pm 1$-methods (see~"FactorsPminus1",
"FactorsPplus1") are only suitable for finding factors with certain
properties and CFRAC (see~"FactorsCFRAC") is just a slower
predecessor of the MPQS). All absolute timings are given for a
Pentium 200 under Windows as a reference machine.

%%%%%%%%%%%%%%%%%%%%%%%%%%%%%%%%%%%%%%%%%%%%%%%%%%%%%%%%%%%%%%%%%%%%%%%%%
\Section{Timings for ECM}

The runtime of `FactorsECM' depends mainly on the size of the factors
of the input number.
On average, finding a 12-digit factor of a 100-digit number
takes about 1~min~40~s, finding a 15-digit factor of a 100-digit
number takes about 10~min and finding an 18-digit factor of a 100-digit
number takes about 50~min.
A~general rule of thumb is the following: one digit more increases the
runtime by a bit less than a factor of two.
These timings are very rough, and they may vary by a factor of 10 or more.
You can compare trying an elliptic curve with throwing a couple of dice,
where a success corresponds to the case where all of them show the same
side -- it is possible to be successful with the first trial, but it is
also possible that this takes much longer. In particular, all trials are
independent of one another.
In general, ECM is superior to Pollard's Rho for finding factors with at
least 10 decimal digits. In the same time needed by Pollard's Rho
for finding a 13-digit factor one can reasonably expect to find a
17-digit factor when using ECM, for which Pollard's Rho in turn would
need  about 100 times as long as ECM.
For larger factors this difference grows rapidly.
From theory it can be said that finding a 20-digit factor requires
about 500 times as much work as finding a 10-digit factor,
finding a 30-digit factor requires about 160 times as much work as
finding a 20-digit factor and finding a 40-digit factor requires
about 80 times as much work as finding a 30-digit factor.

The default parameters are optimized for finding factors with about
15 -- 35 digits. This seems to be a sensible choice, since this is the
most important range for the application of ECM. The function
`FactorsECM' usually gives up when the input number~<n> has two factors
which are both larger than its third root. This is of course only a
``probabilistic'' statement. Sometimes -- but seldom -- the remaining
composite has 3 factors, 4 factors should occur (almost) never.

The user can of course specify other parameters than the default ones,
but giving timings for all possible choices is obviously impossible.
The interested reader should follow the references given in the
bibliography at the end of this manual for getting information on how
many curves with which parameters are usually needed for finding
factors of a given size. This depends mainly on the distribution of
primes, respectively of numbers with prime factors not exceeding a
certain bound.

For benchmarking purposes, the amount of time needed for trying
a single curve with given smoothness bounds for a number of given size
is suited best.
A typical example is the following: one curve with (<Limit1>,<Limit2>) =
(100000,10000000) applied to a 100-digit integer requires a total of
10~min~20~s, where 6~min~45~s are spent for the first stage
and 3 min 35 s are spent for the second stage.
The time needed for the first stage is approximately linear in <Limit1>
and the time needed for the second stage is a bit less than linear
in <Limit2>.

%%%%%%%%%%%%%%%%%%%%%%%%%%%%%%%%%%%%%%%%%%%%%%%%%%%%%%%%%%%%%%%%%%%%%%%%%
\Section{Timings for the MPQS}

The runtime of `FactorsMPQS' depends only on the size of the input
number, and not on the size of its factors.
Rough timings are as follows: 90~s for a 40-digit number, 10~min for a
50-digit number, 2~h for a 60-digit number, 20~h for a 70-digit
number and 100~h for a 75-digit number.
These timings are much more precise than those given for ECM, but they
may also vary by a factor of 2 or 3 depending on whether a good factor
base can be found without using a large multiplier or not.
A general rule of thumb is the following: 10 digits more cause 10 times
as much work. 
For benchmarking purposes, precise timings for some integers are given:
$38!+1$ (45~digits, good factor base with
multiplier~1): 2~min~22~s, $40!-1$ (48~digits, not so good factor
base even with multiplier~7): 8~min~58~s, cofactor of $1093^{33}+1$
(61~digits, good factor base with multiplier~1): 1~h~12~min.

%%%%%%%%%%%%%%%%%%%%%%%%%%%%%%%%%%%%%%%%%%%%%%%%%%%%%%%%%%%%%%%%%%%%%%%%%
%%
%E  timings.tex . . . . . . . . . . . . . . . . . . . . . . . . ends here