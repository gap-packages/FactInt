%%%%%%%%%%%%%%%%%%%%%%%%%%%%%%%%%%%%%%%%%%%%%%%%%%%%%%%%%%%%%%%%%%%%%%%%%
%%
%W  info.tex              FactInt documentation               Stefan Kohl
%%
%H  @(#)$Id$
%%
%%%%%%%%%%%%%%%%%%%%%%%%%%%%%%%%%%%%%%%%%%%%%%%%%%%%%%%%%%%%%%%%%%%%%%%%%

\Chapter{Getting Information about the Factoring Process}

\index{information about factoring process}
\>`IntegerFactorizationInfo' V
\>`InfoFactInt' V

is an Info class (see~"ref:Info Functions" in the {\GAP} Reference Manual
for a description of the Info mechanism) that provides a way of seeing
what's going on during the factoring process.

If `InfoLevel(IntegerFactorizationInfo) = 1', then basic information
about the factoring techniques used is displayed. If this InfoLevel has
value 2, then additionally all ``relevant'' steps in the factoring
algorithms are mentioned, and if it is set to 3, then large amounts of
details of the progress of the factoring process are shown. 

For convenience,

\>FactIntInfo( <level> ) F
\>FactInfo( <level> ) F

sets the `InfoLevel' of `IntegerFactorizationInfo' to the positive integer
<level>. In other words,

\){\kernttindent}FactIntInfo( <level> );

is equivalent to:

\){\kernttindent}SetInfoLevel( IntegerFactorizationInfo, <level> );

The informational output is not guaranteed to be literally the same in
each factorization attempt to a given integer with given parameters.

The timings in the example are given for a Pentium 200.

\begintt
gap> FactIntInfo(2);
gap> Factors(1459^60-1);
#I  
#I  Check for n = b^k +/- 1
#I  
697136005439518321339364341462731479643931164718338822169985761598533184155968\
687683267309071831767783053294661533309653344501590527857911683460652704454392\
2114173712945384179913214273324400 = 1459^60 - 1
#I  The factors corresponding to polynomial factors are
[ 1458, 1460, 2127223, 2128682, 2130141, 4528179181405, 4531280671081, 
  4534390675481, 20518450806493943781446161, 20532514165758816624282961, 
  20546596816316767296268561, 
  421584732115767299535558510031830342788459718258721 ]
#I  Intermediate result : [ [ 4531280671081, 20546596816316767296268561 ], 
  [ 1458, 1460, 2127223, 2128682, 2130141, 4528179181405, 4534390675481, 
      20518450806493943781446161, 20532514165758816624282961, 
      421584732115767299535558510031830342788459718258721 ] ]
#I  
#I  Factors already found : [ 4531280671081, 20546596816316767296268561 ]
#I  
#I  Trial division by all primes p < 1000
#I  Intermediate result : [ [ 2, 3, 3, 3, 3, 3, 3 ], [  ] ]
#I  Intermediate result : [ [ 2, 2, 5, 73 ], [  ] ]
#I  Intermediate result : [ [ 7, 303889 ], [  ] ]
#I  Intermediate result : [ [ 2, 1064341 ], [  ] ]
#I  Intermediate result : [ [ 3, 13, 193, 283 ], [  ] ]
#I  Intermediate result : [ [ 5, 11, 521, 158024051 ], [  ] ]
#I  Intermediate result : [ [ 31, 146270666951 ], [  ] ]
#I  Intermediate result : [ [ 61 ], [ 336368046008097439040101 ] ]
#I  
#I  Factors already found : [ 2, 2, 2, 2, 3, 3, 3, 3, 3, 3, 3, 5, 5, 7, 11, 
  13, 31, 61, 73, 193, 283, 521, 303889, 1064341, 158024051, 146270666951, 
  4531280671081, 20546596816316767296268561 ]
#I  
#I  Trial division by some already known primes
#I  
#I  Check for perfect powers
#I  
#I  Pollard's Rho
Steps = 16384, Cluster = 1638
Number to be factored : 
336368046008097439040101
#I  Intermediate result : [ [ 2251, 149430495783250750351 ], [  ] ]
#I  
#I  Pollard's Rho
Steps = 16384, Cluster = 1638
Number to be factored : 
20532514165758816624282961
#I  
#I  Pollard's Rho
Steps = 16384, Cluster = 1638
Number to be factored : 
421584732115767299535558510031830342788459718258721
#I  
#I  Factors already found : [ 2, 2, 2, 2, 3, 3, 3, 3, 3, 3, 3, 5, 5, 7, 11, 
  13, 31, 61, 73, 193, 283, 521, 2251, 303889, 1064341, 158024051, 
  146270666951, 4531280671081, 149430495783250750351, 
  20546596816316767296268561 ]
#I  
#I  Pollard's p - 1
Limit1 = 10000, Limit2 = 400000
Number to be factored : 
20532514165758816624282961
#I  Initializing prime differences list, PrimeDiffLimit = 1000000
#I  p-1 for n = 20532514165758816624282961
a : 2, Limit1 : 10000, Limit2 : 400000
#I  First stage
#I  Second stage
#I  
#I  Pollard's p - 1
Limit1 = 10000, Limit2 = 400000
Number to be factored : 
421584732115767299535558510031830342788459718258721
#I  p-1 for n = 421584732115767299535558510031830342788459718258721
a : 2, Limit1 : 10000, Limit2 : 400000
#I  First stage
#I  Second stage
#I  
#I  Williams' p + 1
Residues = 2, Limit1 = 2000, Limit2 = 80000
Number to be factored : 
20532514165758816624282961
#I  p+1 for n = 20532514165758816624282961
Residues : 2, Limit1 : 2000, Limit2 : 80000
#I  Residue no. 1
#I  Residue no. 2
#I  
#I  Williams' p + 1
Residues = 2, Limit1 = 2000, Limit2 = 80000
Number to be factored : 
421584732115767299535558510031830342788459718258721
#I  p+1 for n = 421584732115767299535558510031830342788459718258721
Residues : 2, Limit1 : 2000, Limit2 : 80000
#I  Residue no. 1
#I  Residue no. 2
#I  
#I  Elliptic Curves Method (ECM)
Curves = <func.>
Init. Limit1 = <func.>, Init. Limit2 = <func.>, Delta = <func.>
Number to be factored : 
20532514165758816624282961
#I  ECM for n = 20532514165758816624282961
#I  Digits : 26, Curves : 3
#I  Initial Limit1 : 200, Initial Limit2 : 20000, Delta : 200
#I  Curve no.      1 (     3), Limit1 :     200, Limit2 :    20000
#I  Curve no.      2 (     3), Limit1 :     400, Limit2 :    40000
#I  11-digit factor 12077430061 was found in second stage
#I  Intermediate result : [ [ 12077430061, 1700073116718901 ], [  ] ]
#I  
#I  Elliptic Curves Method (ECM)
Curves = <func.>
Init. Limit1 = <func.>, Init. Limit2 = <func.>, Delta = <func.>
Number to be factored : 
421584732115767299535558510031830342788459718258721
#I  ECM for n = 421584732115767299535558510031830342788459718258721
#I  Digits : 51, Curves : 27
#I  Initial Limit1 : 200, Initial Limit2 : 20000, Delta : 200
#I  Curve no.      1 (    27), Limit1 :     200, Limit2 :    20000
#I  Curve no.      2 (    27), Limit1 :     400, Limit2 :    40000
#I  Curve no.      3 (    27), Limit1 :     600, Limit2 :    60000
                             [ ... ]
#I  Curve no.     25 (    27), Limit1 :    5000, Limit2 :   500000
#I  Curve no.     26 (    27), Limit1 :    5200, Limit2 :   520000
#I  Curve no.     27 (    27), Limit1 :    5400, Limit2 :   540000
#I  
#I  Factors already found : [ 2, 2, 2, 2, 3, 3, 3, 3, 3, 3, 3, 5, 5, 7, 11, 
  13, 31, 61, 73, 193, 283, 521, 2251, 303889, 1064341, 158024051, 
  12077430061, 146270666951, 4531280671081, 1700073116718901, 
  149430495783250750351, 20546596816316767296268561 ]
#I  
#I  Multiple Polynomial Quadratic Sieve (MPQS)
Number to be factored : 
421584732115767299535558510031830342788459718258721
#I  MPQS for n = 421584732115767299535558510031830342788459718258721
#I  Digits                     :         51
#I  Multiplier                 :          1
#I  Size of factor base        :       1326
#I  Prime powers to be sieved  :       1349
#I  Length of sieving interval :     131072
#I  Small prime limit          :        117
#I  Large prime limit          :    3500186
#I  Number of used a-factors   :          4
#I  Size of a-factors pool     :         52
#I  Initialization time        :         11 sec.
#I  
#I  Sieving
#I  
#I  Complete factorizations over the factor base   :       50
#I  Relations with a large prime factor            :        5
#I  Relations remaining to be found                :     1343
#I  Total factorizations with a large prime factor :      614
#I  Used polynomials                               :       67
#I  Elapsed runtime                                :       39 sec.
#I  Progress (relations)                           :        3 %
#I  
                             [ ... ]
#I  
#I  Complete factorizations over the factor base   :      651
#I  Relations with a large prime factor            :      826
#I  Relations remaining to be found                :        0
#I  Total factorizations with a large prime factor :     9164
#I  Used polynomials                               :     1021
#I  Elapsed runtime                                :      455 sec.
#I  Progress (relations)                           :      100 %
#I  
#I  Creating the exponent matrix
#I  Doing Gaussian Elimination, #rows = 1477, #columns = 1378
#I  Processing the zero rows
#I  Dependency no. 1 yielded no factor
                 [ ... ]
#I  Dependency no. 8 yielded no factor
#I  Dependency no. 9 yielded factor 26725378467920336641
#I  The factors are
[ 26725378467920336641, 15774696422795817479975816558881 ]
#I  Digit partition : [ 20, 32 ]
#I  MPQS runtime : 473.507 sec.

#I  Intermediate result : 
[ [ 26725378467920336641, 15774696422795817479975816558881 ], [  ] ]
#I  
#I  The result is
[ [ 2, 2, 2, 2, 3, 3, 3, 3, 3, 3, 3, 5, 5, 7, 11, 13, 31, 61, 73, 193, 283, 
      521, 2251, 303889, 1064341, 158024051, 12077430061, 146270666951, 
      4531280671081, 1700073116718901, 26725378467920336641, 
      149430495783250750351, 20546596816316767296268561, 
      15774696422795817479975816558881 ], [  ] ]

#I  The total runtime was 799.408 sec

[ 2, 2, 2, 2, 3, 3, 3, 3, 3, 3, 3, 5, 5, 7, 11, 13, 31, 61, 73, 193, 283, 
  521, 2251, 303889, 1064341, 158024051, 12077430061, 146270666951, 
  4531280671081, 1700073116718901, 26725378467920336641, 
  149430495783250750351, 20546596816316767296268561, 
  15774696422795817479975816558881 ]
\endtt

%%%%%%%%%%%%%%%%%%%%%%%%%%%%%%%%%%%%%%%%%%%%%%%%%%%%%%%%%%%%%%%%%%%%%%%%%
%%
%E  info.tex  . . . . . . . . . . . . . . . . . . . . . . . . . ends here
